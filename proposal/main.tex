\documentclass[12pt]{article}
\usepackage[spanish]{babel}
\usepackage{geometry}

\title{Application for a Study Trip \\ for Colombian
  Students\\---\\ First Colombo-German \\ Astronomy Winter School 2013} 
\author{Jaime E. Forero-Romero, PhD \\ {\small Assistant Professor, Universidad de los Andes, Bogot\'a, Colombia}\\
{\small on behalf of the {\textit{Red Virtual de Estudios en Ciencias Astrof\'isicas}} (REVECA)}}
\date{30.04.2013}

\begin{document}

\maketitle
This is an application for a Study Trip to Germany from 1.12.2013 till
12.12.2013. The objective is to take a group of bright Colombian students to
leading German institutes in the area of Astronomy in order to take lectures,
develop small research projects and get a clear picture of what it
takes to pursue graduate studies in Germany. 

This application is done under the A-Modality, whereby the DAAD takes
most of the responsibility for the logistics of the trip. However, we
are willing to make the trip under the B-Modality where we take full
responsibility to make all the logistic arrangements. The trip will
focused on three cities: Potsdam, Tautenburg and Heidelberg. In total
seven different Universities/Institutes will be visited.

The student group is composed by the 15 most promising young
astronomers in Colombia. They come from seven different universities
and five different cities. The most important selection criteria were
academic excellence, promise to be great researchers and passion for
astronomy.  

In the next section we present a detailed justification for this trip,
an itinerary and the preparation process before the trip. Attached as
an annex are the invitation letters from German institutions and the
letters of support from the Colombian institutions.  

\newpage

\section{Justification}

Astronomy is a key scientific area for development. It attracts young
minds to the study of math, sciences and engineering. It is an area
that ultimately tries to answer deep questions related to our
cosmic origins while making ground-breaking progress to build new
telescopes, instruments and detectors.

Colombia is a country where this area of knowledge is starting to develop
at Universities. Since 2008 there are national meetings such as the
Colombian Congress for Astronomy and Astrophysics (COCOA), which has
held three meetings ever since in Medellin, Bogot\'a and Bucaramanga
with a strong attendance of 100+ people including researchers, students and guests. During the
last COCOA 2012 in Bucaramanga, we realized that developing Astronomy
in Colombia crucially depends on building stronger bonds of
collaboration among all the researchers in the country and giving the
best opportunities to the youngest generation. Organizing a school
with students from all over the country in Germany seemed to be a
great way to reach that goal. 

With scientific contacts in Potsdam, Heidelberg and Tautenburg we
decided to focus the school in those cities. The common theme is
Extragalactic Astrophysics, given that its already one of the
strongest research areas in Colombia. 

{\bf School's philosophy and structure}. The main objective of the school is
two-fold. First, get the Colombian students to interact as much as possible with
graduate students and researchers in Germany. Second, give the
Colombian students the opportunity to get first
hand information on how to pursue graduate studies in Germany.  This
is why we have structured the mornings with presentations highlighting
the science done at each institute and the structure of graduates
schools in Germany. The afternoons are devoted to work on small
projects together with PhD students and postdocs. There will also be
two opportunities (in Potsdam and Heidelberg) where the
students will give short talks to present the scientific results of
their work.

{\bf About the student group}. With the above mentioned objectives we
decided to look for the most promising undergrad students in
Colombia.  We look for three elements in the students: academic
excellence, potential to be researchers and passion for astronomy. We
also required that the students should have a good command of spoken
and written English.

With these requirements we ended up with an heterogeneous group of
students. All of them are between the third and tenth semester of
their undergraduate careers. They all have experience with small
research projects and a solid knowledge of basic subjects 
(mechanics, electromagnetism, calculus and programming) needed to
get the most out of the school. Most of them have also spent
time abroad in summer schools.  There are a few that are ready to
start their masters in 2014. 

Nevertheless, as a preparation to attend the school we will hold a
seminar on basic subjects of Extragalactic Astronomy\footnote{This is
  explained in detailed in Section   \ref{sec:preparation} of this
  document}. 


{\bf About the professor leading the group}. Jaime E. Forero-Romero
is an astrophysicist and assistant professor in the Physics Department
at the Universidad de los Andes in Bogota (Colombia). He studied
Physics at the \'Ecole Normale Sup\'rieure in Paris and holds a PhD in
Physics given by the \'Ecole Normale Supérieure de Lyon. He lived four 
years in Berlin while being a postdoc (Wissenschaftlicher Mitarbeiter)
at the Leibniz Institute for Astrophysics. In 2011 he received the Gruber Fellowship by the
International Astronomical Union to spend one year in the Astronomy
Department at UC Berkely as a postdoctoral researcher in the galaxy
formation group. He still keeps an active scientific collaboration
with members of the Cosmology Group in Potsdam.

{\bf Professors and Colombian Universities supporting this application}. This application was
possible with the help of many other astronomy researchers in
Colombia, who performed the student selection and helped to shape this
proposal. They are: Giovanni Pinz\'on and Gregorio
Portilla (Observatorio Astron\'omico Nacional), Luis N\'u\~nez and
Jerson Reina (Universidad Industrial de Santander), C\'esar Valenzuela
(Universidad del Valle), Edilberto S\'anchez (Universidad Distrital
Francisco Jos\'e de Caldas), Nelson Vera (Universidad Pedag\'ogica y Tecnol\'ogica de
Colombia) and Jorge Zuluaga (Universidad de Antioquia). 


\section{Preparation to the School}
\label{sec:preparation}

As a preparation for the trip, the Postgraduate Seminar of the Astronomy
Group at Universidad de los Andes will be open to all the students participating in the
Winter School. This seminar will be lead by the Prof. Forero-Romero. 

For the students living in Bogot\'a (half of the students in the group)
it will be possible to meet in person on a weekly basis at Universidad
de los Andes to attend the Seminar. The students outside Bogot\'a
will work remotely with their tutors in each city, so they can follow the same
themes. Once every month we will meet virtually via teleconferences in
order to discuss the advances of that month. The seminar
will be held in English as training for the School. 

This 4-month seminar will have the following time-line.
\begin{itemize}
\item First Month. Present basic concepts on the theme of the School
  (Extragalactic Astronomy)
\item Second Month. Get to know the research subjects of the groups at
  the Institutes to be visited in Germany.
\item Third Month. Pick two (2) of these research subjects to contact
  tutors in Germany to ask for introductory literature.
\item Fourth Month. Develop in depth the beginning of the small
  research projects which will be finished during the visit to Germany. 
\end{itemize}

The mechanics of the seminar will be 20 minute-long presentations by the
Prof. Forero-Romero, visiting scientists and the students. We will follow the
scheme used during the second semester of 2012 in the Undergraduate
Astronomy seminar at Universidad de los Andes\footnote{See this
  webpage for more details:
  \verb"http://forero.github.io/AstroSeminarUniandes/"}. 




\section{Itinerary}

\begin{tabular}{p{2cm}p{2cm}p{2.5cm}p{7.0cm}}
Day & Date & Time & Activity\\\hline\hline

Sunday & 1.12.2013 & 8PM&	Travel from Bogot\'a to Berlin\\\hline

Monday &2.12.2013& 3PM & Arrival to Berlin\\\hline

Tuesday &3.12.2013& 9AM-11AM & Scientific Talks at the Leibniz Institute for Astrophysics (AIP)\\
 & & 11AM-12AM &Tour around AIP\\
 & & 1PM-5PM & Work with tutors (AIP)\\\hline

Wednesday &4.12.2013& 9AM-11AM &Scientific Talks at the Max Planck
for Gravitation (AEI)\\
 & & 11AM-12AM &Tour around AEI\\
 & & 1PM-5PM & Work with tutors (AIP/AEI)\\\hline

Thursday &5.12.2013 &  9AM-10AM & Talks: Graduate Studies in Potsdam (AIP)\\
& &  10AM-12AM & Tour of Telegrafensberg's AIP site\\
& & 1PM-5PM& Work with tutors (AIP/AEI)\\ \hline

Friday &6.12.2013& 9AM-12M & Work with tutors (AIP/AEI) \\		
      & & 2PM-5PM&  Students' presentations (AIP)\\\hline

Saturday & 7.12.2013& 8AM-3PM&	Train to Jena - Bus to Tautenburg\\
 & &4PM-7PM & Tour of all the observational facilities\\
 & &7PM-12PM & If weather permits: Observing Practice\\\hline

Sunday &8.12.2013& 9AM-2PM & Bus to Jena - Train to Heidelberg\\\hline

Monday &9.12.2013  & 9AM-11M & Scientific Talks at Haus der
Astronomie (HdA)\\ 
&  & 11AM-12M  & Tour around the Max Planck Institut for Astronomy (MPIA)\\
 & & 13PM-17PM & Work with tutors (HdA)\\\hline

Tuesday &10.12.2013  & 9AM-11AM & Talks: graduate studies in Heidelberg (HdA)\\
& & 11AM-13M & Tour around Zentrum fuer Astronomie (ZAH) at Heidelberg Universitaet.\\
 & & 13PM-17PM & Work with tutors (HdA)\\\hline

Wednesday &11.12.2013 & 9AM-12AM	&Work with tutors (HdA)\\
& & 2PM-4PM  & Students' presentations (HdA)\\\hline

Thursday &12.12.2013&11AM	& Travel from Heidelberg to Bogot\'a\\\hline\hline
\end{tabular}



\section{Invitation Letters from Germany}
Invitation letters from the following German institutions are included:

\begin{itemize}
\item Leibniz-Institut fuer Astrophysik Potsdam
\item Haus der Astronomie (Heidelberg)
\item Max Planck Institut fuer Astronomie (Heidelberg)
\item Max Planck Institut fuer Gravitationsphysik (Potsdam)
\item Thueringer Landessternwarte Tautenburg
\end{itemize}

\newpage

\section{Support Letters from Colombia}
Support letters from the following Colombian institutions are included:
\begin{itemize}
\item Universidad de los Andes (Bogot\'a)
\item Observatorio Astron\'omico Nacional (Bogot\'a)
\item Universidad de Antioquia (Medell\'in)
\item Universidad Distrital Francisco Jos\'e de Caldas (Bogot\'a)
\item Universidad Industrial de Santander (Bucaramanga)
\item Universidad del Valle (Cali)
\item Universidad Pedag\'ogica y Tecnol\'ogica de Colombia (Tunja)
\end{itemize}


\end{document}
