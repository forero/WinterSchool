\documentclass[12pt]{article}
\usepackage[spanish]{babel}
\usepackage{geometry}

\title{Application for a Study Trip \\ for Colombian
  Students\\---\\ First Colombo-German \\ Astronomy Winter School 2013} 
\author{Jaime E. Forero-Romero, PhD \\ {\small Assistant Professor}
  \\ {\small Universidad de los Andes, Bogot\'a, Colombia}}
\date{30.04.2013}

\begin{document}

\maketitle
This is an application for a Study Trip to Germany from 1.12.2013 till
12.12.2013. The objective is to take a group of bright Colombian students to
leading German institutes in the area of Astronomy in order to take lectures,
develop small research projects and have a clear picture of what it
takes to pursue graduate studies in Germany. 

This application is done under the A-Modality, whereby the DAAD takes
most of the responsibility for the logistics of the trip. However, we
are willing to make the trip under the B-Modality where we take full
responsability to make all the logistic arrangements. The trip will
focused on three cities: Heidelberg, Potsdam and Tautenburg. In total
seven different Universities/Institutes will be visited.

The group is composed by the 15 most promising young astronomers in
Colombia. They come from seven different universities and five different
cities. The most important selection criteria were academic excellence,
promise to be great researchers and passion for astronomy.

In the next section we present a detailed justification for this trip,
a detailed itinerary and the preparation process before the
trip. Attached as an annex are the Invitation Letters from Germany and
the Letters of Support from Colombia.

\newpage

\section{Jusification}

The students were selected by each University

\section{Itinerary}
The itinerary is the following

\begin{tabular}{p{2cm}p{2cm}p{2.5cm}p{7.0cm}}
Day & Date & Time & Activity\\\hline\hline
Sunday & 1.12.2013 & &	Flight leaves from Bogot\'a \\\hline
Monday & 2.12.2013 & &	Arrival to Frankfurt. Travel to Heidelberg.\\\hline
Tuesday &3.12.2013  & 8AM-10M & Scientific Talks at Haus der
Astronomie (HdA)\\
&  & 10AM-12M  & Tour around the Max Planck Institut forAstronomy (MPIA)\\
 & & 13PM-17PM & Work with tutors (HdA)\\\hline
Wednesday &4.12.2013  & 8AM-10AM & Talks: graduate studies in Heidelberg (HdA)\\
 & & 10AM-12M  &Second Tour around MPIA\\
 & & 13PM-17PM & Work with tutors (HdA)\\\hline
Thursday &5.12.2013&8AM-10AM	&Morning: Scientific Talks (HdA)\\
& & 10AM-12M & Tour around Zentrum fuer Astronomie (ZAH) at Heidelberg Universitaet.\\
& & 13PM-17PM  & Work with tutors (HdA)\\\hline

Fr &6.12.2013&&	Morning: Student Presentations (HdA)   		Afternoon: Flight/Train to Berlin\\

Sa &7.12.2013&&	Morning: Train to Jena - Bus to Tautenburg   		Afternoon: Tour of all the facilities\\
Su &8.12.2013&	&Morning: Bus to Jena - Train to Berlin.   		Afternoon: Free - Move to new hotel in Potsdam\\
Mo &9.12.2013&	&Morning: Scientific Talks and Tour (Leibniz
Institut-AIP)	Afternoon: Work with tutors (AIP)\\ 

Tu &10.12.2013&	&Morning: Scientific Talks and Tour (Max Planck Inst. -
Uni Potsdam)   		Afternoon: Work with tutours (AIP).\\ 

We &11.12.2013 &  &Morning: Work with tutors (AIP)
Afternoon: Student's talks\\ 

Th &12.12.2013&	& Travel from Berlin to Bogota\\
\end{tabular}



\section{Preparation}

As a preparation to the trip the Postgraduate seminar of the Astronomy
group will be open to all the students participating in the
schoool. For the students living in Bogota (half of the students in
the group) we will meet in person on a weekly basis at Universidad de
los Andes. For the students of the outside bogota we will work with
their tutors in each city so they can follow the same themes as in
Bogota. Once every month we will meet virtually via teleconferences in
order to discuss the advances and themes of that month. The seminar
will be held in English as training for the school. 

This 4-month seminar will have the following timeline.
\begin{itemize}
\item First Month Present basic concepts on the theme of the School
  (Extragalactic Astronomy)
\item Second Month. Get to know the research subjects of the groups at
  the Institutes to be visited in Germany.
\item Third Month. Pick two (2) of these reserach subjects to contact
  the tutours in Germany to ask for introductory lietaratyre.
\item Fourth Month. Develop in depth the beginnig of the small
  research projects to be finished during the visit to Germany. 
\end{itemize}

The mechanics of the seminar will be 20 minute-long presentations by the
Lecturer, visiting scienctists and the students. We will follow the
scheme used during the second semester of 2012 in the Undergraduate
Astronomy seminar at Universidad de los Andes\footnote{See this webpage for more details: http://forero.github.io/AstroSeminarUniandes/}.




\section{Invitation letters from Germany}

\section{Support Letter from Colombia}


\end{document}
