\documentclass[12pt]{article}
\usepackage[spanish]{babel}
\usepackage{geometry}

\title{Application for a Study Trip for Colombian Students}
\author{Jaime E. Forero-Romero, PhD \\ {\small Assistant Professor, Universidad de los Andes, Bogot\'a, Colombia}}
\date{30.04.2014}

\begin{document}

\maketitle
This is an application for a Study Trip to Germany from 4.1.2015 till
16.01.2015. The objective is to take a group of bright Colombian students to
leading German institutes in the area of Astronomy in order to take lectures,
develop small research projects and get a clear picture of what it
takes to pursue graduate studies in Germany. 

This application is done under the B-Modality, where we take full
responsibility to make all the logistic arrangements. However, we are
willing to make the trip under the A-Modality whereby the DAAD takes
most of the responsibility for the logistics of the trip.  The trip will
focus on two cities: Potsdam and Heidelberg. In total  five different
Universities/Institutes will be visited. 
 
The student group is composed by the 10 undergraduate Physics students at
Universidad de lso Andes in Bogot\'a, Colombia. All of them have an
interest to pursue a research carreer in astronomy; half of them have
already tackled small research projects and presented their results in
international venues. 

In the next section we present a detailed justification for the trip
together with the itinerary and the preparation process before
travelling.. Attached as an annex are the invitation letters from
German institutions.

\newpage

\section{Justification}

Astronomy is a key scientific area for development. It attracts young
minds to the study of math, sciences and engineering. It is an area
that ultimately tries to answer deep questions related to our
cosmic origins while making ground-breaking progress to build new
telescopes, instruments and detectors.

Colombia is a country where this area of knowledge is starting to develop
at Universities. Since 2008 there are national meetings such as the
Colombian Congress for Astronomy and Astrophysics (COCOA). There have
been three of such meetings ever since (in Medellin, Bogot\'a and Bucaramanga)
with a strong attendance of 100+ people including researchers,
students and guests. The research community now hosts close to 40
researchers with permanent positions at different Universities.

In order to educate the next generation of researchers I ave decided
to prepare this proposal to take 10 physics students to Potsdam and
Heidelberg, cities that host research institutes and universities of
great quality. The objective of the visit is two-fold

\begin{itemize}
\item
 Give the Colombian students the opportunity to get first hand
 information on how to pursue graduate studies in Germany. 
\item 
  Get the Colombian students to  interact as much as possible with
  graduate students and researchers in Germany.  
\end{itemize}


With this objectives in mind we have structured our visit as
follows. In the mornings we will have presentations highlighting
the science done at each institute and the structure of graduates
schools in Germany; in the afternoons the students will to work on small
projects together with PhD students and postdocs. There will also be
two opportunities (in Potsdam and Heidelberg) where the students will
give short talks to present theresults of their work.

We will also have a preparation before the tripo. We will hold a
seminar on different subjects of astronomy and astrophysics that have
been chosen of interest by the studens and that are covered by the
research groups in Potsdam and Heidelberg. This will be done in the
framework of the Undergraduate Astronomy Seminar at Universidad de los
Andes \footnote{The webpage for the current semester can be seen here:
\texttt{http://forero.github.io/AstroSeminarUniandes/}}


{\bf About the professor leading the group}. Jaime E. Forero-Romero
is an astrophysicist and assistant professor in the Physics Department
at the Universidad de los Andes in Bogota (Colombia). He studied
Physics at the \'Ecole Normale Sup\'rieure in Paris and holds a PhD in
Physics given by the \'Ecole Normale Supérieure de Lyon. He lived four
years in Berlin working as a postdoc (Wissenschaftlicher Mitarbeiter)
at the Leibniz Institute for Astrophysics. In 2011 he received the
Gruber Fellowship by the International Astronomical Union to spend one
year in the Astronomy Department at UC Berkely as a postdoctoral
researcher in the galaxy formation group. He still keeps an active
scientific collaboration with members of the Cosmology Group in
Potsdam. In June 2014, sponsored by DAAD, he will make a research
visit to the Theoretical Astrophysics Group in the Heidelberg
Insitute for Theoretical Studies. 
 

\section{Preparation to the Trip}
\label{sec:preparation}

As a preparation for the trip, the Undergraduate Seminar of the Astronomy
Group at Universidad de los Andes will be open to all the students
participating in the trip. This seminar will be lead by the
Prof. Forero-Romero.  

During the first month the students will explore the research subjects of the groups at  the Institutes to be visited in Germany. Each student will pick a
  research subject of interest to contact tutors in Germany and get a
  small research project assigned.  In the next three months the
  students will develop the projects under the supervision of
  Prof. Forero-Romero. By the end of the seminar they will present the
  advances of their work and the possibilities to finish/extend the
  work during the trip to Germany.



\section{Itinerary}

\begin{tabular}{p{2cm}p{2cm}p{2.5cm}p{7.0cm}}
Day & Date & Time & Activity\\\hline\hline

Monday & 5.1.2015 & 8PM&	Travel from Bogot\'a to Berlin\\\hline

Tuesday & 6.1.2015 & 5PM & Arrival in Berlin. \\\hline

Wednesday & 7.1.2015 & 9AM-11AM & Scientific Talks at the Leibniz Institute for Astrophysics (AIP)\\
 & & 11AM-12M &Tour around AIP\\
 & & 1PM-5PM & Work with tutors (AIP)\\\hline

Thursday &8.1.2015& 9AM-11AM &Scientific Talks at the Max Planck
for Gravitation (AEI)\\
 & & 11AM-12M &Tour around AEI\\
 & & 1PM-5PM & Work with tutors (AIP/AEI)\\\hline

Friday &9.1.2015 &  9AM-10AM & Talks: Graduate Studies in Potsdam (AIP)\\
& &  10AM-12M & Tour of Telegrafensberg's AIP site\\
& & 1PM-5PM& Work with tutors (AIP/AEI)\\ \hline

Saturday & 10.1.2015&  &	Free day in Berlin\\\hline

Sunday & 11.1.2015 & 2PM-8PM & Train from Berlin Hbf to Heidelberg Hbf\\\hline

Monday & 12.1.2015  & 9AM-11M & Scientific Talks at Haus der Astronomie (HdA)\\ 
&  & 11AM-12M  & Tour around the Max Planck Institut for Astronomy (MPIA)\\
 & & 13PM-17PM & Work with tutors (HdA)\\\hline

Tuesday &13.1.2015  & 9AM-11AM & Talks: graduate studies in Heidelberg (HdA)\\
& & 11AM-12PM & Tour around Zentrum fuer Astronomie (ZAH) at Heidelberg Universitaet.\\
 & & 13PM-17PM & Work with tutors (HdA)\\\hline

Wednesday &14.11.2015 & 9AM-11AM	& Scientific Talks at the Heidelberg
Insitute for Theoretical Studies (HITS) \\
&  & 11AM-12M  & Tour around HITS\\
& & 2PM-5PM  & Student's presentations (HdA)\\\hline

Thursday &15.1.2015& 9AM-4PM	& Train from Heidelberg Hbf to Berlin Hbf
\\\hline

Friday &16.1.2015&12M & Plane from Berlin to Bogot\'a \\\hline\hline


\end{tabular}



\section{Invitation Letters from Germany}
Invitation letters from the following German institutions are included:

\begin{itemize}
\item Leibniz-Institut fuer Astrophysik (Potsdam)
\item Max Planck Institut fuer Gravitationsphysik (Potsdam)
\item Haus der Astronomie (Heidelberg)
\item Max Planck Institut fuer Astronomie (Heidelberg)
\item Heidelberg Center for Theoretical Studies (Heidelberg)
\end{itemize}

\section{Support Letter from Universidad de los Andes}
The suport letter by the Dean of the Faculty of Sciences is included.

\section{Students}

\begin{tabular}{lll}
Names & Last Names & Semester \\
Maria Camila & Remolina Gutierrez & \\
Maria Fernanda & Gomez Alvarez & \\ 
Paulina & Hoyos Restrepo& \\
Sebastian & Caldas Rivera & \\
Sebastian & Velasco Moreno & \\
Diego Edison & Uma\~na & \\
Santiago & Aguirre Lamus & \\
German Felipe & Giraldo Villa & \\
Daniel Esteban & Ochoa Tamayo & \\
\end{tabular}

\end{document}
